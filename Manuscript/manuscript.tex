\documentclass[]{article}
\usepackage{lmodern}
\usepackage{amssymb,amsmath}
\usepackage{ifxetex,ifluatex}
\usepackage{fixltx2e} % provides \textsubscript
\ifnum 0\ifxetex 1\fi\ifluatex 1\fi=0 % if pdftex
  \usepackage[T1]{fontenc}
  \usepackage[utf8]{inputenc}
\else % if luatex or xelatex
  \ifxetex
    \usepackage{mathspec}
  \else
    \usepackage{fontspec}
  \fi
  \defaultfontfeatures{Ligatures=TeX,Scale=MatchLowercase}
\fi
% use upquote if available, for straight quotes in verbatim environments
\IfFileExists{upquote.sty}{\usepackage{upquote}}{}
% use microtype if available
\IfFileExists{microtype.sty}{%
\usepackage[]{microtype}
\UseMicrotypeSet[protrusion]{basicmath} % disable protrusion for tt fonts
}{}
\PassOptionsToPackage{hyphens}{url} % url is loaded by hyperref
\usepackage[unicode=true]{hyperref}
\hypersetup{
            pdfborder={0 0 0},
            breaklinks=true}
\urlstyle{same}  % don't use monospace font for urls
\usepackage{longtable,booktabs}
% Fix footnotes in tables (requires footnote package)
\IfFileExists{footnote.sty}{\usepackage{footnote}\makesavenoteenv{long table}}{}
\usepackage[normalem]{ulem}
% avoid problems with \sout in headers with hyperref:
\pdfstringdefDisableCommands{\renewcommand{\sout}{}}
\IfFileExists{parskip.sty}{%
\usepackage{parskip}
}{% else
\setlength{\parindent}{0pt}
\setlength{\parskip}{6pt plus 2pt minus 1pt}
}
\setlength{\emergencystretch}{3em}  % prevent overfull lines
\providecommand{\tightlist}{%
  \setlength{\itemsep}{0pt}\setlength{\parskip}{0pt}}
\setcounter{secnumdepth}{0}
% Redefines (sub)paragraphs to behave more like sections
\ifx\paragraph\undefined\else
\let\oldparagraph\paragraph
\renewcommand{\paragraph}[1]{\oldparagraph{#1}\mbox{}}
\fi
\ifx\subparagraph\undefined\else
\let\oldsubparagraph\subparagraph
\renewcommand{\subparagraph}[1]{\oldsubparagraph{#1}\mbox{}}
\fi

% set default figure placement to htbp
\makeatletter
\def\fps@figure{htbp}
\makeatother


\date{}

\begin{document}

FUSTr: a tool to find gene Families Under Selection in Transcriptomes

\textbf{Background}

Recent advances in RNA-Seq technologies have allowed for an abundance of
protein coding sequence data to be generated across all levels of
biodiversity {[}1--4{]}. In non-model eukaryotic study systems,
transcriptomic experiments have become the de facto approach for
functional genomics in lieu of whole genome resequencing. This is due
largely in part to lower costs {[}5{]}, better targeting of coding
sequences {[}6{]}, exploration of posttranscriptional modifications and
differential gene expression {[}7,8{]}, \sout{These experiments have
allowed research in functional genomics to be extended to non-model
organisms with a broad range of phenotypic diversity ~(i.e.
morphological, behavioral, physiological, etc.) ~that has never before
been possible.} This influx of transcriptomic data has resulted in an
ever-expanding need for scalable tools capable of \sout{elucidating
patterns and processes involved in the adaptive evolution of genes and
genomes of organisms throughout the tree of life.}

It is important to be able toelucidating patterns and processes involved
in the adaptive evolution of genes and genomes of \emph{organisms} in
order to understand the vast phenotypic diversity found in nature.
~Speciation events \sout{forming gene orthologs} along with frequent
whole genome duplications \sout{forming gene paralogs} has given rise to
a myriad of multigene families that span a broad range of biochemical
properties. There are several families adaptive and fitness familes
contain genes that contribute to organisms fitness and adaptive in
\emph{ways.} with \emph{various} biochemical properties that contribute
the vast phenotypic diversity\sout{which has contriuted to the vast
phenotypic diversity found across all domains of life} (i.e
~{[}9--11{]}.

\begin{itemize}
\item
\item
  \begin{quote}
  phenotypic diversity ~(i.e. morphological, behavioral, physiological,
  etc.)
  \end{quote}
\item
  \begin{quote}
  elucidating patterns and processes involved in the adaptive evolution
  of genes and genomes of organisms throughout the tree of life.
  \end{quote}
\end{itemize}

Grouping protein encoding genes into their respective families de novo
has remained a difficult task computationally \sout{that has been shown
to be an NP-hard problem.} This typically entails homology searches in
large amino acid sequence similarity networks with graph partitioning
algorithms in order to cluster coding sequences into \emph{transitive}
groups {[}12--14{]}\emph{.} This is further complicated in eukaryotic
transcriptome datasets that contain several isoforms via alternative
splicing, which cannot be treated as phylogenetically independent
homologs, \emph{more words} {[}\textbf{??{]}}. Further analysis of these
gene families is also non-trivial, as it requires multiple sequence
alignment followed by phylogenetic inference\sout{, both of which has
been demonstrated to be NP-complete problems} {[}15--18{]}. Further
exploration of patterns of molecular evolution in these families is also
computationally intensive, requiring robust phylogenetic analysis using
codon substitution models with random or mixed effects likelihood
methods in addition to MCMC Bayesian statistical frameworks, in order to
determine patterns of pervasive diversifying selection or episodic
lineage based diversifying selection {[}19, 20{]}.

~~~~~~~ Here we present FUSTr, a tool to address the aforementioned
difficulties of characterizing molecular evolution in large biodiversity
transcriptomic datasets in a pipeline capable of scaling to multicore
high-performance computational facilities. FUStr can be used to
characterizing selective regimes on homologous groups of
phylogenetically independent coding sequences in transcriptomic datasets
and has been verified using Arachnoserver and simulated datasets. The
presented pipeline implements simplified user experience with minimized
third-party dependencies, in an environment robust to breaking changes
to maximize reproducibility over a long-term time scale. ~

FUSTr is freely available under a GNU license and can be downloaded at
\href{https://github.com/tijeco/Fuster}{\emph{https://github.com/tijeco/Fuster}}

\emph{\textbf{P4}}

\begin{itemize}
\item
  Here we present things that do stuff

  \begin{itemize}
  \item
    Deal with complexities in the previous stated issues
  \end{itemize}
\item
  What it is
\item
  What it does
\item
  Why its useful?
\item
  How it performs/accuracy
\item
  Where it is
\end{itemize}

\textbf{Implementation}

FUSTr is available as a Snakefile using snakemake, to ensure scalbility
and reproducibility. In order to further increase the reproducibility,
all necessary dependencies have been wrapped up in an Anaconda
environment. ~

\textbf{Pre-filter data}

Input data consists of assembled transcriptome fasta files, nucleotide
data. These files are first filtered to remove sequences just containing
Ns as well as to removing any haphazard text found in sequences (that
may be artifacts of previous assembling procedures) to ensure proper
downstream analysis. Header patterns in the inputs are simulteneosly
autodetected to sort out all unique and redundant identifiers for
downstream isofom filtration.

\begin{longtable}[]{@{}lll@{}}
\toprule
gene & AOXIE & id=a\tabularnewline
\midrule
\endhead
gene & 111 & id=b\tabularnewline
gene & 2342 & id=a\tabularnewline
gene & 2342 & id=b\tabularnewline
gene & 2342 & id=c\tabularnewline
\bottomrule
\end{longtable}

\begin{itemize}
\item
  Algorithms, bitches and money, layout of typical gene and isoform
  header patterns
\end{itemize}

\textbf{Predict coding sequences}

Coding sequences are determined using Transdecoder, which predicts orfs
using all 6 possible reading frames, keeping the best orf per
transcript. Genes containing several isoforms will only keep longest
isoform for downstream stuff. protein sequences, stuff. Keep only
longest isoform.

\textbf{Group by homology}

Sequence homology networks are determined using an all against all BLAST
search of the amino acid sequences. Sequences are grouped into
homologous groups using Silix, only adding sequences to a group that
have at least \#\#\# sequence coverage, \#\#\# sequence identity,
\&\&\&\& other stuff. All against all blast. Silix. Families.

\textbf{Multiple-sequence alignment and phylogenetic inference}

Homologous groups containing at least 15 amino acid sequences are
aligned using Mafft (accurate and fast, auto), and then trimmed using
trimal. Fastree is used to infer phylogenetic tree (accurate and fast)
Mafft, trimal, codon masking. Codon sequences are masked over amino acid
alignment.

\textbf{Pervasive positive selection}

Codon alignments. CODEML, pervasive positive selection.Codeml and
FUBAR??

\sout{There are several established phylogenetic frameworks to then test
the families for various forms of selection utilizing ~rates of
nonsysnonymous to synonymous substitution. Determining selective regimes
of specific amino acid \emph{residue} sites that \emph{may be adaptive}
~involves tests of pervasive positive selection within a gene family
using either fixed or random effects likelihood models to help
ellucidate specific amino acid residue sites undergoing diversifying or
purifying selection (FUBAR, M8). Finding specific lineages that may have
undergone \emph{adaptive surges} (niche diffenertiation, sexual
selection, predator prey arms races, novel innovations, lots of
evolution terms) requires tests for episodically diversifying lineages
(MEME,BUSTED,CodemlBranchSpecific) or evolution along specific branches
in gene family\ldots{} }

\textbf{Episodic positive selection}

Hyphy/MEME, FUBAR, BUSTED

\textbf{Validation}

Simulations, GLOve??? Bacteria stuff (POTION), and arachnoserver
(compare to young clade paper).

Results

\textbf{Simulations}

We did stuff, yeah.

\textbf{Empirical results}

Arachnoserver, toxinbase, bacteriashit, goldenstandard.

\textbf{Conclusions}

\sout{An influx of \emph{terabytes} of transcriptomic data has resulted
in an ever expanding need for scalable} tools capable of elucidating
broad patterns of molecular evolution within the genomic architecture of
taxa spanning throughout the tree of life.

\end{document}
